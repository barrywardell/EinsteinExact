\documentclass{article}

\usepackage{../../../doc/latex/cactus}

\begin{document}

\author{Barry Wardell and Ian Hinder}
\title{EinsteinExact}
\date{10 May 2012}

\maketitle

% Do not delete next line
% START CACTUS THORNGUIDE

\begin{abstract}
The \textsc{EinsteinExact} arrangement provides a collection of thorns which compute
exact solutions to the Einstein equation and set the $3+1$ ADM variables
provided by the ADMBase thorn. The solutions may be computed either as initial
data, or at each iteration of an evolution. The computation is done analytically
from expressions for the metric which are taken from a database of solutions
provided by the \textsc{Metrics} package for \textsc{Mathematica}. Adding new
solutions to \textsc{EinsteinExact} only requires that the metric be added to this
database.
\end{abstract}

\section{Introduction}
The \textsc{EinsteinExact} arrangement provides a collection of thorns which compute
exact solutions to the Einstein equation and set the $3+1$ ADM variables
provided by the ADMBase thorn. The solutions may be computed either as initial
data, or at each iteration of an evolution. The computation is done analytically
from expressions for the metric which are taken from a database of solutions
provided by the \textsc{Metrics} package for \textsc{Mathematica}. Adding new
solutions to \textsc{EinsteinExact} only requires that the metric be added to this
database.

Each thorn sets the ADM variables for a specific spacetime in some
particular coordinate system. It therefore only makes sense to have one
thorn in the arrangement activated at a time.

\section{$3+1$ Decomposition of the metric}
The \textsc{EinsteinExact} code relies on the fact that a given analytic
expression for the $4$-metric $g_{ab}$ (and its inverse $g^{ab}$) of a spacetime
may be decomposed into the $3+1$ ADM variables,
\begin{equation}
g_{ab} =
\left(
\begin{array}{cc}
 -\alpha^2 + \beta_k \beta^k & \beta_i \\
 \beta_i & \gamma_{ij}\\
\end{array}
\right),\quad
g^{ab} =
\left(
\begin{array}{cc}
 -1/\alpha^2 & \beta^i / \alpha^2 \\
 \beta^i / \alpha^2 & \gamma^{ij} - \beta^i \beta^j / \alpha^2\\
\end{array}
\right),
\end{equation}
where $\alpha$ is the lapse, $\beta^i$ is the shift and $\gamma_{ij}$
is the three metric on the slice. The \textsc{ADMBase} variables for the
lapse, shift and three metric may therefore be directly obtained from
components of the four metric and its inverse:
\begin{equation}
\alpha = \frac{1}{\sqrt{-g_{00}}},\quad
\beta^i = \alpha^2 g^{0i},\quad
\gamma_{ij} = g_{ij}.
\end{equation}
Another \textsc{ADMBase} variable which must be set is the extrinsic
curvature $K_{ij}$. This may be written as
\begin{equation}
K_{ij} = -\frac{1}{2\alpha} \mathcal{L}_{\alpha \vec{n}} \gamma_{ij}
\end{equation}
where $\mathcal{L}$ is the Lie derivative and $\vec{n}$ is the normal to the
slice. Explicitly expanding this out gives
\begin{align}
K_{ij} = & \frac{1}{2\alpha} (-\partial_t + D_i \beta_j + D_j \beta_i) \nonumber \\
 = &  \frac{1}{2\alpha} (-\partial_t
        + \beta^k \partial_k \gamma_{ij}
        + \gamma_{ik} \partial_j \beta^k
        + \gamma_{kj} \partial_i \beta^k
      )
\end{align}
which is readily computed from the previous analytic expressions for $\alpha$,
$\beta^i$ and $\gamma_{ij}$. Finally, the \textsc{ADMBase} variables for the time
derivatives for the lapse and shift are computed by analytically differentiating
$\alpha$ and $\beta^i$ with respect to $t$.

\section{Obtaining EinsteinExact}
The latest version of \textsc{EinsteinExact} will always be available from the
project git repository:

\code{git clone --recursive git://github.com/barrywardell/EinsteinExact}

\section{Using EinsteinExact}
To use \textsc{EinsteinExact}, include the thorn for your desired spacetime in your
thornlist and in the \code{ActiveThorns} section of your parameter file and set either
the \code{ADMBase::initial\_data} or \code{ADMBase::evolution\_method}
parameters described below.

\subsection{Common Parameters}
Although there is a separate Cactus thorn for each supported spacetime, all
thorns are generated from the same Kranc script and therefore share some
common parameters. The parameters supported by all thorns are listed in this
section.

\subsubsection{\code{ADMBase::initial\_data}}
The ADM variables can be set as initial data for an evolution. Each
EinsteinExact thorn extends the \code{ADMBase::initial\_data}
keyword parameter with another option corresponding to the name of the thorn.
If the parameter is set to the thorn name, then initial data for the ADM
variables will be set by the thorn in the \code{ADMBase\_InitialData} schedule
bin.

\subsubsection{\code{ADMBase::evolution\_method}}
The setting of the ADM variables can optionally be done at each step in an
evolution. Each EinsteinExact thorn extends the \code{ADMBase::evolution\_method}
keyword parameter with another option corresponding to the name of the thorn.
If the parameter is set to the thorn name, then the ADM variables will be set
by the thorn at each iteration during the \code{ANALYSIS} schedule bin.

\subsubsection{\code{ADMBase::initial\_lapse}}
Each EinsteinExact thorn extends this keyword parameter with another option
corresponding to the name of the thorn. This parameter is currently ignored
by the code.

\subsubsection{\code{ADMBase::initial\_shift}}
Each EinsteinExact thorn extends this keyword parameter with another option
corresponding to the name of the thorn. This parameter is currently ignored
by the code.

\subsubsection{\code{ADMBase::initial\_dtlapse}}
Each EinsteinExact thorn extends this keyword parameter with another option
corresponding to the name of the thorn. This parameter is currently ignored
by the code.

\subsubsection{\code{ADMBase::initial\_dtshift}}
Each EinsteinExact thorn extends this keyword parameter with another option
corresponding to the name of the thorn. This parameter is currently ignored
by the code.

\subsubsection{Origin: \code{positiont}, \code{positionx}, \code{positiony}, \code{positionz}}
These parameters specify the position at which the origin of the spacetime
should be located.

\subsubsection{Shift: \code{shiftx}, \code{shifty}, \code{shiftz}}
Shift that should be added to the metric.

\subsubsection{Rotation: \code{theta}, \code{phi}, \code{psi}}
The spacetime may have an arbitrary $3$D rigid rotation applied to it. This
rotation is specified by the \code{theta}, \code{phi} and \code{psi} parameters
which correspond to the Euler angles of the rotation. The conventions used are
such that the rotations are given by the three rotation matrices:
\begin{align}
R_\phi =& 
\left(
\begin{array}{ccc}
 \cos (\phi ) & \sin (\phi ) & 0 \\
 -\sin (\phi ) & \cos (\phi ) & 0 \\
 0 & 0 & 1 \\
\end{array}
\right), \nonumber \\
R_\theta =&
\left(
\begin{array}{ccc}
 1 & 0 & 0 \\
 0 & \cos (\theta ) & \sin (\theta ) \\
 0 & -\sin (\theta ) & \cos (\theta ) \\
\end{array}
\right), \nonumber \\
R_\psi =&
\left(
\begin{array}{ccc}
 \cos (\psi ) & \sin (\psi ) & 0 \\
 -\sin (\psi ) & \cos (\psi ) & 0 \\
 0 & 0 & 1 \\
\end{array}
\right).
\end{align}

\section{Included Spacetimes}

\subsection{Gauge Wave}

\subsection{Kerr-Schild}

\subsection{Minkowski}

\subsection{ShiftedGaugeWave}

\subsection{Vaidya}

\section{Modifying EinsteinExact}

\subsection{Generating the Code}

\subsection{Adding a spacetime}

\subsection{Updating the Metrics Database}

\section{Acknowledgments}


% END CACTUS THORNGUIDE

\end{document}
